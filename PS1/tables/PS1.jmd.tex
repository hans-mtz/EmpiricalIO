\input webmac
---
title : Empirical IO
subtitle: PS1
author : Hans Martinez
date: 16th May 2020
---

# Introduction


This an example of a julia script that can be published using
[Weave](http://weavejl.mpastell.com/dev/usage/).
The script can be executed normally using Julia
or published to HTML or pdf with Weave.
Text is written in markdown in lines starting with "`#'` " and code
is executed and results are included in the published document.


Notice that you don't need to define chunk options, but you can using
`#+`. just before code e.g. `#+ term=True, caption='Fancy plots.'`.
If you're viewing the published version have a look at the
[source](FIR_design_plots.jl) to see the markup.


<!-- this setup dependencies, but doesn't appear in the generated document -->
```julia; echo = false; results = "hidden"
using Pkg
"Plots" ∉ keys(Pkg.project().dependencies) && Pkg.add("Plots")
"DSP" ∉ keys(Pkg.project().dependencies) && Pkg.add("DSP")
```


# FIR Filter Design


We'll implement lowpass, highpass and ' bandpass FIR filters. If
you want to read more about DSP I highly recommend [The Scientist
and Engineer's Guide to Digital Signal
Processing](http://www.dspguide.com/) which is freely available
online.


## Calculating frequency response


DSP.jl package doesn't (yet) have a method to calculate the
the frequency response of a FIR filter so we define it:

```julia
using Plots, DSP
gr()

function FIRfreqz(b::Array, w = range(0, stop=π, length=1024))
    n = length(w)
    h = Array{ComplexF32}(undef, n)
    sw = 0
    for i = 1:n
      for j = 1:length(b)
        sw += b[j]*exp(-im*w[i])^-j
      end
      h[i] = sw
      sw = 0
    end
    return h
end
```

## Design Lowpass FIR filter


Designing a lowpass FIR filter is very simple to do with DSP.jl, all you
need to do is to define the window length, cut off frequency and the
window. We will define a lowpass filter with cut off frequency at 5Hz for a
signal
sampled at 20 Hz.
We will use the Hamming window, which is defined as:
$w(n) = \alpha - \beta\cos\frac{2\pi n}{N-1}$, where $\alpha=0.54$ and $%
\beta=0.46$

```julia
fs = 20
f = digitalfilter(Lowpass(5, fs = fs), FIRWindow(hamming(61)))
w = range(0, stop=pi, length=1024)
h = FIRfreqz(f, w)
```

## Plot the frequency and impulse response


The next code chunk is executed in term mode, see the [script](FIR_design.jl)
for syntax.

```julia; term=true
h_db = log10.(abs.(h));
ws = w/pi*(fs/2)
```

```julia
plot(ws, h_db,
      xlabel = "Frequency (Hz)", ylabel = "Magnitude (db)")
```

And again with default options

```julia
h_phase = unwrap(-atan.(imag.(h),real.(h)))
plot(ws, h_phase,
    xlabel = "Frequency (Hz)", ylabel = "Phase (radians)")
```



\inx
\fin
\con
